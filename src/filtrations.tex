% !TeX root = ../main.tex

\paragraph{Filtrations}

Given a topological space $X$ a \textbf{filtration} on $X$ is a nested sequence of subspaces $\emptyset= X_0\subseteq X_1\subseteq\ldots \subseteq X_n = X$.
The \textbf{$k$th persistent homology module} of a filtration $\X$ on a topological space $X$ is defined as the persistence module $\H_k(\X)$ consisting of vector spaces $\hom_k(X_i)$ and linear maps $\hom_k(X_i)\to\hom_k(X_j)$ induced by inclusions $X_i\hookrightarrow X_j$ for $i\leq j$.
We will use the notation $\H(\X)$ to refer to the persistent homology of $\X$ in all dimensions $k$.
In particular, given a real-valued function $f: X\to\RR$, the \textbf{sub level set filtration} $\F$ of $f$ is the nested sequence of sub level sets $F_s := f^{-1}((-\infty,s])$.
When convenient, we will refer to the persistent homology of a sub level set filtration $\H(\F)$ as the persistent homology $\H(f)$ of the function itself.

\paragraph{Filtrations on simplicial complexes}

Given a simplicial complex $K$ with $n$ vertices $V$ any function $\varphi : V\to\RR$ let $\varphi^I : K\to\RR$ be defined for $\sigma\in K$ as $\varphi^I(\sigma) = \max_{v\in\sigma} \varphi(v)$ so that the sub level sets $\Phi^I_s$ of $\varphi^I$ are subcomplexes of $K$.
The resulting filtration is referred to as the \textbf{induced filtration} $\Phi^I = \{\Phi^I_s\}_{s\in\RR}$ of $\varphi$ on $K$.
We can also define a function $\varphi^L : |K|\to\RR$ by extending $\varphi$ linearly over the geometric simplices of $\underline{K}$.
That is, for any $x \in |K|$, let $\varphi^L(x) = \sum_{v\in V} x_v f(v)$.
The \textbf{piecewise linear filtration} $\Phi^L = \{\Phi^L_s\}_{s\in\RR}$ of $\varphi$ is defined as the sequence of sub level sets $\Phi^L_s$ of $\varphi^L$, each a subspace of $|K|$ in $\RR^n$.

Because each sub level set $\Phi_s^I$ of the induced filtration is a subcomplex of $K$, the geometric realizations $|\Phi_s^I|$ are subspaces of $|K|$.
In fact, the $|\Phi_s^L|$ are equal to the sub level sets of the piecewise constant extension $\varphi^C : |K|\to \RR$, defined for $x\in |K|$ as $\varphi^C(x) = \max\{\varphi(v)\mid x_v > 0\}$.
Because $\varphi^C \geq \varphi^L$, $|\Phi_s^I|\subseteq \Phi_s^L$ for all $s\in\RR$.

\begin{theorem}\label{thm:induce_linear_iso}
  Let $K$ be a simplicial complex and let $\varphi : V\to\RR$ be a function on its vertices.
  The persistent homology modules of the induced and piecewise linear filtrations $\H(\Phi^I)$ and $\H(\Phi^L)$ of $\varphi$ on $K$ and $|K|$, respectively, are naturally isomorphic with respect to maps induced by inclusions.
\end{theorem}
\begin{proof}
  We will first show that the inclusion $|\Phi^I_s|\hookrightarrow \Phi^L_s$ is a homotopy equivalence that commutes with inclusions $|\Phi^I_s|\hookrightarrow |\Phi^I_t|$ and $\Phi^L_s\hookrightarrow \Phi^L_t$ for all $s\leq t\in\RR$.\textbf{TODO}
  Because each $|\Phi^I_s|$ is the polytope of $\Phi^I_s$, the result will then follow from the equivalence of singular and simplicial homology.

  For any $s\in\RR$ let $V_s = \{ v\in V\mid \varphi(v)\leq s\}$ and define the affine projection of $x\in |K|$ as
  \[ \pi(x) := \frac{\sum_{v\in V_s} x_v \underline{v}}{\sum_{v\in V_s} x_v}.\]
  For any $x\in |\Phi_s^I|$, $\pi(x) = x$ as $x_v = 0$ for all $v\in V\setminus V_s$.
  Moreover, because $\pi(x)$ is a convex combination of vertices in $V_s$, $\varphi^L\circ \pi(x)\leq s$ for all $x\in\Phi_s^L$.
  We can therefore construct a deformation retraction $\Phi^L_s\to |\Phi^I_s|$ as the linear homotopy between the identity and $\pi$
  \[ \gamma(x, t) := (1-t)x + t\pi(x).\]
  Because $\varphi^L$ is linear, we have
  \[ \varphi^L\circ \gamma(x,t) = (1-t)\varphi^L(x) + t\varphi^L\circ \pi(x)\leq s,\]
  so $\gamma(x,t)\in\Phi_s^L$ for all $x\in \Phi_s^L$.
  It follows that the inclusion $|\Phi^I_s|\hookrightarrow \Phi^L_s$ is a homotopy equivalence for all $s\in\RR$.
\end{proof}

% The following lemma states that the

\begin{corollary}\label{cor:induce_sublevel_iso}
  Let $K$ be a triangulation of $X$ with vertex set $V$ and let $f : X\to \RR$.
  If there exists a homeomorphism $h : |K|\to X$ such that $f\circ h$ is the linear extension of $\varphi : V\to \RR$ to $|K|$, then the persistent homology modules of the induced filtration $\H(\Phi^I)$ of $\varphi$ on $K$ and that of the sub level set filtration $\H(\F)$ of $f$ on $X$ are naturally isomorphic with respect to maps induced by inclusions.
\end{corollary}
\begin{proof}
  Because $f\circ h$ is the linear extension of $\varphi$ to $|K|$, $\Phi^L$ is equal to the sub level set filtration of $f\circ h$.
  Because $h$ is a homeomorphism, $\hom_*(\Phi_s^L)$ is naturally isomorphic to $\hom_*(F_s)$ for all $s\in\RR$.
  The result therefore follows by composition with the isomorphism provided by Theorem~\ref{thm:induce_linear_iso}.
\end{proof}
